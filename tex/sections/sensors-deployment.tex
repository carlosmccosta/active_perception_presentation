\section{\scshape Sensors deployment}
\subsection*{Sensors deployment}
\begin{frame}{Sensors deployment}
	\begin{itemize}
		\item For making the estimation of the sensor configuration computational feasible, the 3D continuous space was populated with a given set of sensors that were looking at a given point (with the sensor roll either 0º or random)
		\item Several populations of sensors can be added to the world
		\item Each population is of a given sensor type and is deployed within a given region of interest 
		\begin{itemize}
			\item This allows to restrict the spatial distribution of the sensors, for example a given set of sensors should only be deployed in the walls or ceiling due to their weight, or it should be very close to the target object given its limited depth measurements range
		\end{itemize}
		\item Currently supported deployment configurations:
		\begin{itemize}
			\item Uniform or random deployment within a box
			\item Uniform or random deployment within a cylinder
			\item Uniform within a 2D grid (with a given set of rows and columns)
			\item Uniform along a line
		\end{itemize}
	\end{itemize}
\end{frame}

\begin{frame}{Sensors deployment}
	\begin{itemize}
		\item For the active perception environment, the sensors were deployed close to the target object, on the top, right and back side of the trolley
		\item This was done to simulate the closest range in which a dynamically moving sensor attached to a robotic arm could move (taking into consideration the human safety and the sensor minimum measurement distance, that was 0.2 meters)
	\end{itemize}
	\begin{figure}
		\centering
		\includegraphics[height=.25\textwidth]{sensor-deployment/active-perception/gazebo-front}
		\includegraphics[height=.25\textwidth]{sensor-deployment/active-perception/gazebo-top}
		\caption{Sensors deployment on the active perception environment}
	\end{figure}
\end{frame}

\begin{frame}{Sensors deployment}
	\begin{itemize}
		\item For the single bin picking environments, given that the target object was inside the stacking box, the sensors were deployed close to the target object, but only on top of the trolley, on 3 layers (each with a different type of sensor)
	\end{itemize}
	\begin{figure}
		\centering
		\includegraphics[height=.4\textwidth]{sensor-deployment/bin-picking/gazebo-sensors}
		\includegraphics[height=.4\textwidth]{sensor-deployment/bin-picking-with-occlusions/gazebo-sensors}
		\caption{Sensors deployment on the single bin picking environments}
	\end{figure}
\end{frame}

\begin{frame}{Sensors deployment}
	\begin{itemize}
		\item For the multiple bin picking environment, given that there were multiple target objects (1 inside the stacking box, 1 on top and 2 on the shelves of the trolley), it was deployed 7 populations of sensors:
		\begin{itemize}
			\item 5 layers simulating fixed sensors on the walls and ceiling
			\item 2 layers above the trolley, simulating dynamic sensors attached to a robotic arm
		\end{itemize}
	\end{itemize}
	\begin{figure}
		\centering
		\includegraphics[height=.3\textwidth]{sensor-deployment/multiple-bin-picking-with-occlusions/gazebo-front}\hspace{2em}
		\includegraphics[height=.3\textwidth]{sensor-deployment/multiple-bin-picking-with-occlusions/gazebo-top}
		\caption{Sensors deployment on the multiple bin picking environment}
	\end{figure}
\end{frame}
