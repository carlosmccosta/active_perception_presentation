\section{\scshape Introduction}\label{sec:introduction}

\subsection{Context}
\begin{frame}{Context}
	\begin{itemize}
		\item 3D object recognition is a challenging task that requires active perception of the environment
		\item Simulating the possible static / dynamic configuration of sensors using a set of representative environments can help decide what types of sensors and how many should be used
	\end{itemize}
\end{frame}


\subsection{Research Areas}
\begin{frame}{Research Areas}
	\begin{itemize}
		\item Object recognition
		\item Active perception
		\item 3D simulation and modeling of the environment and sensors
		\item 3D rendering
	\end{itemize}
	
	\begin{itemize}
		{\scriptsize \item Thorsten Gedicke, Martin Günther, Joachim Hertzberg, FLAP for CAOS: Forward-Looking Active Perception for Clutter-Aware Object Search1, IFAC-PapersOnLine, Volume 49, Issue 15, 2016, Pages 114-119, ISSN 2405-8963.}
		{\scriptsize \item R. Saegusa, L. Natale, G. Metta and G. Sandini, "Cognitive robotics - active perception of the self and others," 2011 4th International Conference on Human System Interactions, HSI 2011, Yokohama, 2011, pp. 419-426.}
	\end{itemize}
\end{frame}

\subsection{Software dependencies}
\begin{frame}{Software dependencies}
	\begin{itemize}
		\item Robot Operating System (ROS)
		\item Gazebo simulator
		\item Point Cloud Library (PCL)
	\end{itemize}
	\begin{figure}[!ht]
		\centering
		\includegraphics[height=.08\textheight]{ros-logo}
		\hspace{0.5em}
		\includegraphics[height=.08\textheight]{gazebo-logo}
		\hspace{0.5em}
		\includegraphics[height=.08\textheight]{pcl-logo}
		\caption{Software dependencies logos}
	\end{figure}
\end{frame}
